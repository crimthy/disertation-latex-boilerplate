\documentclass{lib/styles/default-style}

\begin{document}
\pagestyle{no-numbered}
\unnumberedSection{TABLES}

\initTable{MyTable}
% use as transparent?
\addTableVariable{MyTable}{some realy important data & some realy important data & one \\}
\addTaeVariable{MyTable}{some realy important data & data & one \\}
\addTableVariable{MyTable}{some realy important data & end & one \\}
\addTableVariable{MyTable}{some realy important data & end & one \\ \hline}
\printSimpleTable{MyTable}{|cc|c|}
Another table:
\vspace{1em}

\createStyledTable{
    {Just test&\multicolumn{3}{c|}{Test}},
    {Twi&\multicolumn{3}{c|}{Value}},
    {Three& 1&2&3},
    {Four&1&\multicolumn{2}{c|}{Small}},
    {Four&1&2&2},
    {Next&1&2&2}
}{|l|ccc|}

\subsection*{With caption}

Now check text align Створення  інформаційних  систем  сьогодні  здійснюється  
на  основі сучасних  методологічних  концепцій,
які  успадкували  найважливіші  ідеї класичних методоло



\printTableWithCaption{
    \createTable{
    {One&1&2&3},
    {Twi&4&6&8},
    {Three& 1&2&3},
    {Four&1&2&2},
    {Four&1&2&2},
    {Next&1&2&2}
}
}{
    Some good caption
}

\printTableWithCaption{
        \createStyledTable{
            {Властивості & PoW & PoS & DPoS & PBFT},
            {Тип & ймовірнісно-кінцевий & ймовірнісно-кінцевий & ймовірнісно-кінцевий & абсолютно-кінцевий}
        }{|c|c|c|c|c|}
        }{
            Порівняння протоколів консенсусу
        }

        \comment{
            \multirow{2}{*}{Тип}
            &
            \multirow{2}{*}{ймовірнісно-кінцевий}
            &
            \multirow{2}{*}{ймовірнісно-кінцевий}
            &
            \multirow{2}{*}{ймовірнісно-кінцевий}
            &
            \multirow{2}{*}{ймовірнісно-кінцевий}



            \printTableWithCaption{
                \makeTable{
                    \multirow{2}{*}{Multirow} & 
                    \multicolumn{2}{c}{Multi-column}
                }{
                    
                }{|p{5cm}|p{8cm}|p{5cm}|}
            }{
                Визначення характеристик ідеї проекту
            }{startup-char}




            \printTableWithCaption{
                \makeLongTable{
                    Зміст ідеї & Напрямки застосування & Вигоди для користувача
                }{
                    Децентразівона масштабована платформа для створення блокчейн додатків&
                    1. Розробка децентралізованих додатків&
                    1. Можливість розгортати додатки у децентралізованій мережі \\
                    & & 2. Зменшення витрат ресурсів для створення додатків\\
                    & & 2. Зменшення витрат ресурсів для створення додатків\\
                    & & 2. Зменшення витрат ресурсів для створення додатків\\
                    & & 2. Зменшення витрат ресурсів для створення додатків\\
                }{|X[1,c]|X[1,c]|X[1,c]|}
            }{
                Опис ідеї стартап-проекту
            }{startup-idea-test}
        }
\end{document}

