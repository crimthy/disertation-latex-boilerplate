\documentclass{lib/styles/default-style}

\begin{document}
\pagestyle{default-numbered}
\startPageCountFrom{3}

\tableofcontents

\newpage


\unnumberedSection{ВСТУП}

    З кожним роком кількість інформації непреривно збільшується, а кількість цифрової інформації і поготів,
    та, судячи з усього, ця тенденція не планує припинятися. Із ростом кількості цифрової інформації, зростає і
    кількість ресурсів, необхідних для обробки цієї інформації. Хоча вартість цих ресурсів непреривно зменшується,
    проте великі системи потребують чимало коштів. Системи стають складнішими, необхідно ще більше потужностей для обробки інформації.

    Такі системи складно підтримувати, саме тому і почався пошук альтернативних рішень. В останній час все більше людей дивиться у бік
    децентралізованих рішень. Проте, у існуючих рішень є проблеми із масштабованість та швидкістю транзакцій в тому, чи іншому вигляді.

    Вирішення данних проблем може кардинально змінити укореніле уявлення про системи обробки інформації, та дати поштовх до створення
    більш оптимальних систем.

    Результати можуть бути використані у всіх галузях виробництва, де тим чи іншим чином оброблюється інформація.
    Саме тому обрана проблематика дослідження є вельми актуальною на даний момент.

    Метою даної магістерської дисертації є дослідження концепцій для створення гнучкої платформи для розробки швидких децентралізованих додатків
    та реалізація прототипу платформи. Проаналізувати існуючі платформи, запропонувати алгоритм консенсусу  даних  в  розподілених  системах,
    алгоритм сереалізації даних, механізми для оптимізації використання ресурсів в децентралізованих системах,
    спроектувати  систему  котра  підтримує запропоновані концепції.

    Відповідно до мети роботи необхідно вирішити такі завдання:
    \listDefault{
        \item розглянути поняття про платформи для створення децентралізованих додатків;
        \item описати устрій спроектованої системи для розробки блокчейн додатків;
        \item описати реалізацію запропонованих концепцій;
        \item навести результати випробовування розробленого рішення;
        \item розробити стартап-проект.
    }

    Об'єкт дослідження – платформи для створення децентралізованих додатків.

    Предметом дослідження є концепції побудови децентралізованих додатків, які дозволяють створювати ефективні рішеня.

\startSection{АНАЛІЗ ІСНУЮЧИХ ПЛАТФОРМ ДЛЯ СТВОРЕННЯ ДЕЦЕНТРАЛІЗОВАНИХ ДОДАТКІВ}

\subsection{Ethereum}

    Платформа для створення практично будь-яких децентралізованих
    онлайн-сервісів на базі блокчейна (Đapps), що працюють на базі розумнихĐapps), що працюють на базі розумних), що працюють на базі розумних
    контрактів. Реалізована як єдина децентралізована віртуальна машина. Ідея була
    втілена 30 липня 2015 року. Оскільки Ethereum сильно спрощує і здешевлює
    впровадження блокчейна, його впроваджують як великі гравці, такі як Microsoft, IBM, Acronis,
    Сбербанк, банківський консорціум R3, так і нові стартапи.

    У кінці 2013 року Ethereum запропонував дослідник
    і програміст криптовалют Віталік Бутерін. Розвиток фінансувався через інтернет-краудфандінг, який проходив
    у липні-серпні 2014 року. Потім система вийшла у світ 30 липня 2015 року, при цьому 72 мільйони монет були
    створені "попередньо". Це становить близько 68 відсотків загального обсягу постачання у 2019 році.

    У 2016 році,
    внаслідок експлуатації вразливості в програмному забезпеченні проекту, та подальшого розкрадання ефіру на суму 
    50 мільйонів доларів, Ethereum було розділено на дві окремі блокчейн - нова окрема версія стала Ethereum (ETH), а оригінал продовжувався 
    як Ethereum Classic (ETC).

    Ethereum є дуже потужною платформою розробки через технологію розумних контрактів

    Розумний контракт - це код, який працює на EVM.
    Смарт-контракти можуть приймати та зберігати ефір, дані або комбінацію обох.
    Потім, використовуючи логіку, запрограмовану в договорі, він може поширювати
    цей ефір на інші рахунки або навіть інші смарт-контракти.

    Як приклад, приведемо відомий контракт з Бобом та Алісою (рисунок \ref{img:alice-bob-contract-example}). Аліса хоче найняти Боба, щоб побудувати їй внутрішній дворик,
    і вони використовують контракт для депонування (місце для зберігання грошей, доки умова не буде виконана),
    щоб зберігати свій ефір до остаточної транзакції.

    \printImage[1][0.3]{alice-bob-contract-example}{Приклад виконання контракта Боба та Аліси}

    Розумні контракти пишуться мовою, що називається "Solidity". Solidity має статичний тип і підтримує механізм успадкування,
    бібліотеки та складні,визначені користувачем типи. Синтаксис Solidity схожий на JavaScript.

    Програми, що використовують смарт-контракти для їх обробки, називаються "децентралізованими програмами" або "dapps".
    Користувацькі інтерфейси для цих daaps складаються з відомих мов, таких як HTML, CSS та JavaScript.
    Саму програму можна розмістити на традиційному веб-сервері або на децентралізованій файловій службі, наприклад, Swarm або IPFS.

    Враховуючи переваги блокчейна Ethereum, dapp може бути рішенням для багатьох галузей, включаючи, але не обмежуючись ними:

    \listDefault{
        \item Бухгалтерський облік;
        \item Фінанси;
        \item Логістика;
        \item Нерухомість;
        \item Електронні магазиги (маркетплейси).
    }

\subsection{Hyperledger}

    Комплексний проект розробки блокчейну з відкритим вихідним кодом та пов'язаних з цим інструментів,
    який було розпочато Linux Foundation у грудні 2015 року. Основною метою проекту є підтримка спільної
    розробки мереж з розподіленим реєстром, заснованих на технології блокчейн.
    
    У грудні 2015 року Linux Foundation анонсувала створення проекту Hyperledger.
    Імена компаній-засновників були оголошені у лютому 2016, до яких 29 березня того ж року приєдналися ще
    десять учасників і було затверджено склад ради правління. 29 травня виконавчим директором проекту призначили Брайана Белендорфа.
    
    Метою проекту є посилення міжгалузевої співпраці за допомогою технології блокчейну та мереж з розподіленим реєстром.
    Особлива увага приділяється підвищенню продуктивності та надійності цих систем (у порівнянні з аналогічними криптовалютними розробками),
    аби вони могли використовуватися технологічними, фінансовими та компаніями-постачальниками в масштабах глобальних комерційних оборудок. 
    Проект поєднуватиме незалежні відкриті стандарти та протоколи за допомогою фреймворків для створення специфічних модулей, включно з 
    блокчейнами з власними механізмами досягнення консенсусу та порядком збереження даних, а також ідентифікаційними сервісами, контролем
    доступу та смарт-контрактами. 	Попри чутки, згідно з заявою Брайана Белендорфа, введення та використання власної криптовалюти у проекті
    ніколи не відбудеться.    На початку 2016 проект розпочав розглядати пропозиції щодо створення вихідного коду
    та інших ключових технологічних елементів. Однією з перших пропозицій було поєднати попередні розробки Digital Asset,
    механізм досягнення консенсусу від Blockstream та OpenBlockchain від IBM. Пізніше ця технологія отримала назву Fabric (рисунок \ref{img:hyperledger-struct}).
    
    \printImage[1][0.7]{hyperledger-struct}{Структурна схема платформи Hyperledger Fabric}

    У травні розпочалася розробка мережі з розподіленим реєстром Sawtooth від Intel. 

    12 червня 2017 року проект анонсував готову до промислового використання версію Hyperledger Fabric 1.0 який одразу 
    почав набирати популярність на ринку ICO. Того ж місяця London Stock Exchange Group, спільно з IBM, заявила про початок
    розробки блокчейн-платформи на базі Hyperledger Fabric для випуску цифрових акцій італійських компаній. 
    
    У серпні 2017, компанія Oracle приєдналася до консорціуму Hyperledger і  оголосила про початок розробки власного хмарного блокчейн-сервісу. 
    
    У вересні 2017 Королівський банк Канади почав використовувати Hyperledger для міжбанківських розрахунків з США. 
   
    Hyperledger забезпечує такі функціональні можливості для мережі:
    
    \listDefault{
        \item Управління ідентичністю - Hyperledger Fabric надає послугу посвідчення членства,
        яка керує ідентифікаторами користувачів та аутентифікує всіх учасників мережі.
        Списки контролю доступу можуть використовуватися для надання додаткових рівнів дозволу через авторизацію конкретних мережевих операцій;
        \item Приватність та конфідейційність - Hyperledger дозволяє конкуруючим інтересам бізнесу
        та будь-яким групам, які потребують приватних, конфіденційних транзакцій, співіснувати в одній і тій же дозволеній мережі.
        Приватні канали - це обмежені шляхи обміну повідомленнями, які можна використовувати для
        забезпечення конфіденційності та конфіденційності транзакцій для конкретних підмножин членів мережі.
        Усі дані, включаючи інформацію про трансакції, учасників та каналів, на каналі є невидимими та недоступними
        для будь-яких членів мережі, яким явно не надано доступ до цього каналу;
        \item Ефективна обробка - Hyperledger Fabric призначає мережеві ролі за типом вузла.
        Для забезпечення одночасності та паралелізму в мережі виконання транзакцій відокремлено
        від впорядкування та зобов'язань транзакцій. Виконання транзакцій перед їх
        замовленням дозволяє кожному вузлу однорангових обробляти одночасно кілька транзакцій.
        Це одночасне виконання збільшує ефективність обробки кожного партнера та прискорює доставку транзакцій до служби замовлення;
        \item Chaincode - це визначення програмного забезпечення як активів та інструкція щодо
        транзакцій для зміни активів. Chaincode виконує правила читання або зміни пар ключових
        значень або іншої інформації бази даних стану. Функції Chaincode виконуються в базі даних поточного стану
        і ініціюються через пропозицію транзакцій.
    }
\subsection{Corda}

    Платформа з відкритим вихідним кодом, яка дозволяє підприємствам здійснювати операції безпосередньо
    та у суворій конфіденційності за допомогою інтелектуальних контрактів, зменшуючи витрати на операції
    та ведення записів та оптимізацію бізнес-операції. У світі блокчейн платформ, де всі дані передаються
    всім учасникам, сувора модель конфіденційності Corda дозволяє бізнесу здійснювати операції без проблем.
    R3 поставляє два повністю сумісні дистрибутиви платформи - Corda, безкоштовне завантаження коду,
    доступного на GitHub та Corda Enterprise, комерційній версії, яка пропонує функції та послуги, які налаштовані для сучасних підприємств.

    Основоположним об'єктом в концепції є "державний" об'єкт, який є цифровим документом,
    який реєструє існування, зміст та поточний стан угоди між двома або більше сторонами.
    Він призначений для обміну лише з тими, хто має законну причину бачити це.
    Для забезпечення узгодженості глобальної спільної системи, де не всі дані видно всім учасникам,
    використовують безпечні криптографічні хеші для ідентифікації сторін і даних, а також для зв’язку користувачів з попередніми перетвореннями,
    щоб забезпечити ланцюги походження. Головну базу даних визначають як сукупність незмінних об'єктів стану.

    На рисунку \ref{img:corda-example} зображений приклад, державний об'єкт,
    який представляє депозит у розмірі 100 фунтів стерлінгів у комерційному банку,
    що належить вигаданій судноплавній компанії.
    Державний об'єкт посилається на код договору,
    який регулює його переходи, який, ймовірно,
    буде написаний один раз і повторно використаний величезною кількістю штатів, і може посилатися на хеш тої, що регулює юридичну прозу.

    \printImage[1][0.7]{corda-example}{Схема роботи децентралізованого банку на платформі Corda}

    На відміну від більшості існуючих сьогодні платформ для створення блокчейнів додатків, Corda була побудована з чіткою метою
    запису та забезпечення ділових угод між торговими партнерами.
    Як такий, платформа використовує унікальний підхід до розподілу даних та семантики транзакцій,
    підкреслюючи особливості розподілених баз даних, привабливих для фірм, а саме надійне виконання контрактів в автоматизованому
    та примусовому виконанні.

\subsection{Quorum}

    Quorum - протокол розподіленої бази даних на основі Ethereum, розроблений для забезпечення таких галузей,
    як фінанси, ланцюжок поставок, роздрібна торгівля, нерухомість, тощо з дозволеною реалізацією Ethereum,
    що підтримує конфіденційність транзакцій та контрактів.

    Quorum включає мінімалістичний форк клієнта Go Ethereum (geth) і,
    як такий, використовує роботу, яку розпочала спільнота розробників Ethereum.

    Основними відмінностями Quorum є:

    \listDefault{
        \item Конфіденційність транзакцій та контрактів;
        \item Кілька механізмів консенсусу на основі голосування;
        \item  Мережеве/однорангове управління дозволами;
        \item Більш висока продуктивність;
    }

    На даний момент, Quorum складається з архітектурних компонентів, зображених на рисунку \ref{img:quorum-architecture}

    \printImage[1][0.4]{quorum-architecture}{Схема логічної архітектури Quorum}


\subsection{Висновок}

% TODO: Make conclusion for technologies

\startSection{КОНЦЕПЦІЯ ТА ТЕХНОЛОГІЧНІ ПАРАДИГМИ ПЛАТФОРМИ}

%TODO Blabla about conceptions

\subsection{Програмна віртуальна машина}

    Програмна віртуальна машина (PVM), іноді називається віртуальною машиною програми
    або керованим середовищем виконання (MRE), 
    працює як звичайна програма всередині операційної системи та підтримує єдиний процес.
    Вона створюється при запуску цього процесу і руйнується, коли він закінчується.
    Її мета полягає в тому, щоб створити незалежне від платформи середовище програмування,
    яке відкидує деталі апаратного обладнання або операційної системи і
    дозволяє програмі виконуватися однаково на будь-якій платформі.

    Процес VM забезпечує абстракцію високого рівня
    - мову програмування високого рівня (порівняно з низькорівневою абстракцією ISA системи VM).
    Програмна віртуальна машина реалізуються за допомогою інтерпретатора;
    ефективність, порівнянна зі скомпільованими мовами програмування, може бути досягнута за допомогою компіляції "just in time".


    Цей тип VM набув популярності в мові програмування Java (рисунок \ref{img:hotspot-jvm}), яка реалізована за допомогою віртуальної машини Java.
    Інші приклади включають віртуальну машину Parrot і .NET Framework, яка працює на VM під назвою " Common Language Runtime".
    Усі вони можуть служити шаром абстракції для будь-якої мови комп'ютера.
    
    \printImage[1][0.7]{hotspot-jvm}{Архітектура Hotspot JVM}
    
    Особливим випадком віртуальних машин управління є системи,
    які абстрагуються над механізмами зв'язку (потенційно гетерогенного) комп'ютерного кластеру.
    Такий тип VM складається не з одного програмного процесу, а з одного процесу на фізичній машині в кластері.
    Вони розроблені для полегшення завдання програмування багатопоточних додатків, дозволяючи програмісту зосередитися на алгоритмах,
    а не на механізмах зв'язку, що забезпечуються взаємозв'язком та ОС.
    Вони не приховують факту, що спілкування відбувається, і не намагаються представити кластер як єдину машину.

    На відміну від інших віртуальних машин, ці системи не забезпечують конкретної мови програмування,
    але вбудовані в існуючу мову;
    зазвичай така система забезпечує прив'язку для декількох мов (наприклад, C та Fortran).
    Прикладами є паралельна віртуальна машина (PVM)
    та інтерфейс передачі повідомлень (MPI).
    
    Вони не є строго віртуальними машинами, оскільки додатки, що працюють на них, все ще мають
    доступ до всіх служб ОС і тому не обмежені системою.

    Тож, із зазначених переваг можна виділити:

    \listDefault{
        \item Швидкий час розробки - написання програми не займе багато часу,
        оскільки розробникам не потрібно кодувати функції для окремих платформ та специфікацій;
        \item Безпечний код - керовані режими виконання просувають безпечніший код,
        знімаючи з розробників частину відповідальності за безпеку та управління обладнанням;
        \item Нижчі витрати на розгортання - компонентна архітектура спрощує та швидше розгортає додатки у корпоративному середовищі,
        що характеризується багатьма платформами, пристроями та застарілими системами;
        \item Більш якісне програмне забезпечення - керований час виконання звільняє розробників зосереджуватися
        на бізнес-логіці та коді, специфічних для програми, зменшуючи при цьому кількість помилок кодування;
        \item Агностична платформа - завдяки виконанню часу перекладу між вашим додатком та операційною системою
        ви можете кодувати один раз, дозволяючи клієнтам запускати програму в декількох системах;
        \item Чистота коду - простота функціоналу дозволяє писати модульний код, який можна переробити в нові програми та нові системи.
    }

    Отже, із вище зазначених переваг можна точно сказати, шо віртуальна машина необхідна для реалізації платформи, так як це допоможе
    в реалізації кросплатформеності та створить безпечну середу для виконання криптографічних операції.


\subsection{Протокол серіалізації даних}

    Серіалізація даних - це процес перетворення об'єктів даних (рисунок \ref{img:serealization-example}), що знаходяться у складних структурах даних,
    у потік байтів для цілей зберігання, передачі та розповсюдження на фізичних пристроях.

    Комп'ютерні системи можуть відрізнятися за своєю апаратною архітектурою, ОС,
    механізмами адресації. Внутрішні бінарні представлення даних також відповідно
    змінюються в кожному середовищі. Зберігання та обмін даними між такими різними середовищами
    вимагає нейтрального для платформи та мови формату даних, який розуміють усі системи.

    Після передачі серіалізованих даних з вихідної машини на машину призначення
    здійснюється зворотний процес створення об'єктів із послідовності байтів, що називається десяріалізацією.
    Реконструйовані об'єкти - це клони вихідного об'єкта.

    Вибір формату серіалізації даних для програми залежить
    від таких факторів, як складність даних, потреба в читабельності людини, швидкість
    та обмеження місця для зберігання. XML, JSON, BSON, YAML, MessagePack і protobuf - деякі
    часто використовувані формати серіалізації даних.

    \printImage[1][0.7]{serealization-example}{Узагальнений алгоритм серіалізації}

    Комп'ютерні дані зазвичай організовані в структурах даних, таких як масиви,
    таблиці, дерева, класи. Коли структури даних потрібно зберігати або передавати
    в інше місце, наприклад, через мережу, вони серіалізуються.

    Для простих лінійних даних (число або рядок) нічого робити.
    Серіалізація стає складною для вкладених структур даних та посилань на об'єкти.
    Коли об'єкти вкладені в кілька рівнів, наприклад, у деревах, він згортається на ряд
    байтів, і достатньо інформації (наприклад, порядок обходу) включається для відновлення
    початкової структури дерева на стороні призначення.

    Коли об'єкти із посиланнями вказівників
    на інші змінні члена серіалізовані, посилаються об'єкти відстежуються та
    серіалізуються, забезпечуючи, щоб той самий об’єкт не був серіалізований більше одного разу.
    Однак усі вкладені об'єкти теж повинні бути серіалізованими.

    Нарешті, серіалізований потік даних зберігається у послідовності байтів,
    використовуючи стандартний формат.
    ISO-8859-1 - популярний формат для 1-байтного представлення англійських символів та цифр.
    UTF-8 - світовий стандарт кодування багатомовних, математичних та наукових даних; кожен символ може приймати 1-4 байти даних у Unicode.

    На рисунку \ref{img:serialization-data-example} зображено приклад серіалізації структури у текстовий формат.

    \printImage[1][0.6]{serialization-data-example}{Приклад серіалізації даних}
    
    Якщо визначати, які з форматів для яких цілей підходять краще, то можна визначити, що:

    \listDefault{
        \item Швидкість - бінарні формати швидші (рисунок \ref{img:protobuf-example}), ніж текстові формати.
        Google Protobuf має найкращі показники на сьогодні.
        При стиснених даних різниця швидкостей ще більше.
        Для додатків, які не потребують інтенсивного використання даних або в режимі реального часу,
        JSON є кращим із-за читабельності та відсутності схем;
        \item Розмір даних - це стосується фізичного простору в байтах після серіалізації.
        Для невеликих даних стислі дані JSON займають більше місця в порівнянні з бінарними форматами, як Protobuf.
        При великих файлах розрив звужується. Як правило, бінарні формати завжди займають менше місця;
        \item Корисність - людиночитаємі формати, такі як JSON, природно переважніші перед бінарними форматами.
        Для редагування даних добре підходить YAML.
        Для складних типів даних бібліотеки міжплатформової серіалізації дозволяють визначати структуру даних на схемах
        (для текстових форматів) або IDL (для бінарних форматів).
        Визначення схеми в Protobuf дуже легко, якщо використовувати вбудовані інструменти;
        \item Сумісність, розширюваність - JSON є закритим форматом.
        XML є середнім за версією схеми. Зворотна сумісність (розширювані схеми) найкраще обробляється Protobuf.
    }

    \printImage[1][0.6]{protobuf-example}{Графік, на якому Uber Engineering показує, що Protobuf/Thift із стисненням zlib/Snappy
    пропонує найкращий компроміс між швидкістю та розміром.}

    Навіть у Big Data серіалізація стосується перетворення даних у портативні потоки байтів.
    Але управління схемою - ще одна важлива ціль у Big Data.
    Проблеми з узгодженістю даних, такі як пробіли або неправильні значення даних,
    можуть бути дуже дорогими, залучаючи великі зусилля щодо очищення даних.

    Наступний важливий аспект - можливість легко розділяти
    та реконструювати дані (наприклад, MapReduce).
    JSON або XML можуть не працювати належним чином.
    Apache Hadoop має власний формат серіалізації на основі схеми під назвою Avro (рисунок \ref{img:avro}),
    схожий на Protobuf. Схеми Apache також визначаються на основі JSON.
    Apache Hadoop використовує RPC для розмови з різними компонентами. 

    \printImage[1][0.6]{avro}{Принцип роботи Avro}
    
    Механізм серіалізації є важливою частиною платформи, оскільки не тільки дозволяє досягти агностичності, але і збільшити
    швидкодію та зменшити об’єм сховища даних.

\subsection{Протокол консенсусу}

    Принциповою проблемою розподілених обчислювальних і багатоагентних систем
    є досягнення загальної надійності системи за наявності низки несправних процесів.
    Це часто вимагає від процесів узгодження певної вартості даних, яка потрібна під час обчислення.
    Приклади застосувань консенсусу включають, чи слід здійснювати транзакцію в базі даних, узгоджуючи особу лідера,
    центральну машинну реплікацію (генезис) та атомарну трансляцію.
    
    Однак у процесі застосування технології blockchain виникає багато проблем і питань,
    серед яких велике питання - як розробити відповідний протокол консенсусу.

    Консенсус блокчейна полягає в тому, що всі вузли підтримують однакове розподілене сховище.
    У традиційній архітектурі програмного забезпечення консенсус навряд чи є проблемою через існування центрального сервера,
    отже, інші вузли потрібно лише узгодити з сервером. Однак у розподіленій мережі, такій як блокчейн, кожен вузол є і хостом,
    і сервером, і йому потрібно обмінюватися інформацією з іншими вузлами, щоб досягти консенсусу. Іноді деякі вузли будуть
    працювати в режимі онлайн або в режимі офлайн, а також з’являться деякі шкідливі вузли, що серйозно вплине на систему або знищить
    процес консенсусу. Тому відмінний консенсус-протокол може допустити виникнення цих явищ і мінімізувати шкоду,
    щоб не вплинути на кінцевий результат консенсусу.
    
    Крім того, прийнятий системою протокол консенсусу
    також повинен бути придатним для типу блокчейн, який використовується системою. Загалом, можна виділити три основні типи blockchain:
    загальний блокчейн, консорціумний блокчейн та приватний блокчейн.
    
    Кожен тип blockchain має різні сценарії застосування.
    Таким чином, прийнятий протокол консенсусу повинен відповідати вимогам конкретного сценарію застосування.

    \subsubsection{POW}

    PoW вибирає один вузол, щоб створити новий блок у кожному раунді консенсусу шляхом конкуренції з
    обчислювальної потужності. У змаганні вузлам-учасникам потрібно розв'язати криптографічну проблему.
    Вузол, який першим вирішить задачі, може мати право створити новий блок.
    Потік створення блоку в PoW представлений на рисунку \ref{img:flow-pow}.

    \printImage[1][0.6]{flow-pow}{Cхема роботи POW алгоритму}

    Розв'язати проблему, яку представляє PoW дуже важко.
    Вузлам потрібно постійно коригувати значення nonce, щоб отримати правильну відповідь,
    що вимагає великої обчислювальної потужності.
    Зловмисник може скинути один блок в ланцюжку, але в міру збільшення дійсних блоків ланцюга
    також накопичується навантаження, тому для скидання довгого ланцюга потрібна величезна обчислювальна потужність.
    PoW належить до протоколів консенсусу ймовірнісно-кінцевих, оскільки він гарантує можливу послідовність.


    \subsubsection{POS}

    У PoS вибір кожного раунду вузла, який створює новий блок, залежить від утримуваного кола, а не обчислювальної потужності.
    Хоча вузлам все-таки потрібно вирішити задачу SHA256, представлену формулою \ref{eq:pos_sha}:

    \begin{equation}
        SHA512(timestamp, previous hash...) < target*coin
        \label{eq:pos_sha}
    \end{equation}

    Від PoW відмінність полягає в тому, що вузлам не потрібно багато разів коригувати нуль,
    натомість ключовим для вирішення цієї задачі є кількість ставок (монет).
    Отже, PoS - це енергозберігаючий консенсус-протокол, який використовує спосіб стимулювання внутрішньої валюти,
    а не витрачає багато обчислювальної сили для досягнення консенсусу.
    Схема алгоритму виконання PoS показана на рисунку \ref{img:flow-pos}.
      
    \printImage[1][0.6]{flow-pos}{Cхема роботи POS алгоритму}

    Як і PoW, PoS також є протоколом консенсусу ймовірнісно-кінцевих.
    PPcoin була першою криптовалют, яка застосувала PoS до блокчейн.
    У PPcoin, крім розміру ставки, вік монети також вводиться для вирішення задачі PoS.
    Наприклад, якщо ви тримаєте 10 монет протягом 20 днів, то ваш вік монет - 200.
    Після того, як вузол створить новий блок, його вік монет очиститься до 0.
    Крім PPcoin, багато криптовалют використовують PoS, наприклад , Nxt, Odour.

    \subsubsection{DPoS}

    Принцип DPoS - дозволити вузлам, які мають пакет голосів,
    обирати верифікаторів блоків (тобто, творців блоків).
    Цей спосіб голосування змушує зацікавлені сторони надавати право створювати блоки для делегатів, яких вони підтримують,
    а не створювати блоки, таким чином зменшуючи їх обчислювальне енергоспоживання до нуля.
    Алгоритм роботи DPoS на рисунку \ref{img:flow-dpos}

    \printImage[1][0.6]{flow-dpos}{Cхема роботи DPoS алгоритму}
    
    DPoS - це як парламентська система, якщо делегати не зможуть генерувати блоки в свою чергу,
    вони будуть звільнені, а зацікавлені сторони виберуть нові вузли для їх заміни. DPoS максимально використовує
    голоси акціонерів для досягнення консенсусу справедливим та демократичним способом.
    Порівняно з PoW та PoS, DPoS - консенсус-протокол з низькою вартістю та високою ефективністю
    Існують також деякі криптовалюти, які приймають DPoS, такі як BitShares, EOS.
    Нова версія EOS перетворила DPoS на BFT-DPoS (Byzantine Fault Tolerance-DPoS).

    \subsubsection{PBFT (Practical Byzantine Fault Tolerance)}

    PBFT - це візантійський протокол допуску відмов з низькою складністю алгоритму та високою практичністю в розподілених системах.
    PBFT містить п’ять етапів: запит, попередня підготовка, підготовка, виконання та відповідь. На рисунку \ref{img:flow-pbft}
    зображено, як працює PBFT.

    \printImage[1][0.6]{flow-pbft}{Cхема роботи PBFT алгоритму}

    Первинний вузол пересилає повідомлення, надіслане клієнтом, на три інші вузли.
    У випадку збою 3 вузла одне повідомлення проходить через п'ять фаз, щоб досягти консенсусу серед цих вузлів.
    Нарешті, ці вузли відповідають клієнту для завершення консенсусу.

    PBFT гарантує, що вузли підтримують загальний стан і вживають послідовних дій у кожному раунді консенсусу.
    PBFT досягає мети міцної послідовності, таким чином, це протокол консенсусу абсолютної остаточності.
    
    Новий протокол під назвою Stellar - це поліпшення PBFT.
    Stellar приймає протокол FBA (Федеративна візантійська угода), в якому вузли можуть обрати федерацію,
    якій вони довіряють, щоб провести процес консенсусу.

    \subsubsection{Порівняння алгоритмів консенсусу}

    Для того, щоб обрати, який з алгоритмів буде найкраще підходити для
    даної платформи, необхідно спочатку порівняти їх за певними критеріями (таблиця \ref{tbl:cons-comp}):

    \printTableWithCaption{
        \makeTable{
            Властивості & PoW & PoS & DPoS & PBFT
        }{
            Тип & ймовірнісно-кінцевий & ймовірнісно-кінцевий & ймовірнісно-кінцевий & абсолютно-кінцевий \\
            \hline
            Відмовостійкість & 50\% & 50\% & 50\% & 33\% \\
            \hline
            Споживання енергії & Велике & Низьке & Низьке & Відсутне \\
            \hline
            Масштабованість & Добре & Добре & Добре & Погано \\
            \hline
            Застосування & Публічне & Публічне & Публічне & Приватне \\
        }{|c|p{4cm}|p{4cm}|p{4cm}|p{4cm}|}
    }{
        Порівняння протоколів консенсусу
    }{cons-comp}
    
    Як можна побачити, найоптимальнішого алгоритму для даної платформи не існує, проте PoS добре показує себе у відкритих системах,
    в той час як BPFT може себе показати у приватних блокчейнах.

\subsection{Механізм обробки подій}

    Цикл подій - це програмана модель дизайну, яка чекає і пересилає події або повідомлення в програмі.
    Цикл подій працює (рисунок \ref{img:eloop-example}), надсилаючи запит до якогось внутрішнього або зовнішнього "постачальника подій"
    (який, як правило, блокує запит, поки подія не надійшла), а потім викликає відповідного обробника подій ("розсилає подію").

    \printImage[1][0.5]{eloop-example}{Cхема роботи Event Loop на чотирьох потоках}

    Event-Driven підхід, може відокремлювати потоки від з'єднань,
    які використовують лише потоки для подій на конкретних зворотних викликах або обробниках.

    Керована подіями архітектура складається з творців подій та споживачів подій.
    Творець, який є джерелом події, знає лише, що подія сталася.
    Споживачі - це суб'єкти, які повинні знати, що відбулася подія.
    Вони можуть бути залучені до опрацювання події або можуть просто вплинути на подію.

    Даний підхід дозволить максимально оптимально використовувати ресурси, тим самим підвищуючи швидкість встановлення консенсусу в системі.

\subsection{Механізм зовнішнього доступу}

    Для того, щоб створювати додатки для користувачів, необхідний механізм, який надаватиме змогу отримувати данні системи ззовні,
    таким чином буде можливість підключати різноманітні клієнтські додатки із мінімальними зусиллями.

    \subsubsection{RPC}

    Віддалений виклик процедури (RPC) - це коли комп'ютерна програма спричиняє виконання процедури (підпрограми) в
    іншому адресному просторі (зазвичай на іншому комп’ютері в спільній мережі),
    який кодується так, ніби це був звичайний (локальний) виклик процедури.
    Тобто, програміст пише по суті той самий код, чи підпрограма є локальною для виконавчої програми, або віддаленою.
    Це форма взаємодії клієнт-сервер (абонент - клієнт, виконавець - сервер), що зазвичай реалізується через систему передачі
    повідомлень запит-відповідь (рисунок \ref{img:rpc-example}). У об'єктно-орієнтованій парадигмі програмування виклики RPC
    представлені вилученим методом виклику (RMI).

    \printImage[1][0.7]{rpc-example}{Cхема роботи виклику за допомогою RPC}

    Модель RPC передбачає рівень прозорості місцеположення, а саме те,
    що процедури виклику значною мірою однакові, незалежно від того,
    локальні вони або віддалені, але зазвичай вони не є ідентичними,
    тому локальні виклики можна відрізнити від віддалених.
    Віддалені виклики зазвичай на порядок повільніші і менш надійні, ніж локальні, тому важливо розрізняти їх.

    RPC - це форма міжпроцесорного зв'язку (IPC), оскільки різні процеси
    мають різні адресні простори: якщо на одній хост-машині вони мають чіткі віртуальні
    адресні простори, хоча фізичний простір адрес однаковий; якщо вони знаходяться на різних хостах,
    фізичний адресний простір відрізняється.

    \subsubsection{RESTful}
    
    REST - архітектурний стиль програмного забезпечення, який визначає набір обмежень,
    які будуть використані для створення веб-служб(рисунок \ref{img:restful-example}).
    
    \printImage[1][0.7]{restful-example}{Принцип роботи Restful API}

    "Веб-ресурси" вперше були визначені у Всесвітній павутині як документи або файли, визначені за їх URL-адресами.
    Однак сьогодні вони мають набагато більш загальне та абстрактне визначення, яке охоплює будь-яку річ чи сутність,
    які можна будь-яким чином ідентифікувати, назвати, звертатись чи обробляти в Інтернеті.
    У веб-службі RESTful запити, що надсилаються до URI ресурсу, викликають відповідь з корисним навантаженням,
    відформатованим у HTML, XML, JSON чи іншому форматі. Відповідь може підтвердити, що деякі зміни були внесені
    до збереженого ресурсу, і відповідь може забезпечити гіпертекстові посилання на інші пов'язані ресурси або набори ресурсів.
    Якщо використовується HTTP, як це найбільш часто, доступними операціями (методами HTTP) є GET, HEAD, POST, PUT,
    PATCH, DELETE, CONNECT, OPTIONS та TRACE.
    

    \subsection{Висновок}

% TODO 

\startSection{РЕАЛІЗАЦІЯ ТА ТЕХНІЧНА ОПТИМІЗАЦІЯ}
\subsection{Механізм серіалізації (Dragnit)}
    Dragnit - це бінарний формат на основі схем для ефективної сереалізації дерев даних. Він натхненний
    Google Protobuf, але простіший, проте має більш компактне кодування та кращу підтримку додаткових полів.
    
    \subsubsection{Особливості алгоритму}

    Основні особливості алгоритму:

    \listDefault{
        \item Ефективна серіалізація загальних значень - кодування змінної довжини використовується для числових значень, 
        де малі значення займають менше місця;
        \item Ефективна серіалізація складних об'єктів - структурна функція підтримує вкладені об'єкти з нульовою серіалізацією
        накладних витрат;
        \item Виявлення наявності необов'язкових полів - це неможливо в protobuf, особливо для повторних полів;
        \item Лінійна серіалізація -  читання та запис - це операції одноканального сканування;
        тому вони ефективні в кеш-пам'яті та мають гарантовану часову складність;
        \item Зворотна сумісність - нові версії схеми все ще можуть читати старі дані;
        \item Сумісність вперед -  старі версії схеми можуть необов'язково читати нові дані;
        якщо копія нової схеми в комплекті з даними (нова схема дозволяє декодеру пропускати через невідомі поля);
        \item Проста реалізація: API дуже мінімальний, і згенерований код Rust залежить від одного крейта;
    }

    Серіалізатор підтримує такі типи:

    \listDefault{
        \item Нативні типи;
        \listInnerDefault{
            \item bool - значення, яке зберігає істинне або хибне. Буде використовувати 1 байт;
            \item byte - Непідписане 8-бітове ціле число. Очевидно, використовується 1 байт;
            \item int: ціле 32-бітне ціле значення, що зберігається з використанням кодування;
            змінної довжини, оптимізованого для зберігання чисел з невеликою величиною. Буде використано не більше 5 байт;
            \item uint- ціле 32-бітне ціле значення, що зберігається з використанням кодування;
            змінної довжини, оптимізованого для зберігання малих невід’ємних чисел. Буде використано не більше 5 байт;
            \item float - 32-бітове число з плаваючою комою. Зазвичай використовує 4 байти;
            але для значення нуля використовує 1 байт (денормальні числа стають нульовими при кодуванні);
            \item string - рядок з нульовим завершенням UTF-8. Буде використаний принаймні 1 байт;
            \item T [] - будь-який тип може бути перетворений у масив, використовуючи суфікс [];
        }
        \item Користувацькі типи;
        \listInnerDefault{
            \item enum - uint з обмеженим набором значень,
            які ідентифікуються за назвою.
            Нові поля можна додавати до будь-якого повідомлення, зберігаючи зворотну сумісність;
            \item struct - складене значення з фіксованим набором полів,
            які завжди обов'язкові та записані в порядку.
            Нові поля не можуть бути додані до структури, коли ця структура використовується;
            \item message - складене значення з необов’язковими полями.
            Нові поля можна додавати до будь-якого повідомлення, зберігаючи зворотну сумісність.
        }
    }

    На лістингу \ref{code:simple_schema_dragnit} зображена проста схема для Drargnit.

    \begin{stdout}{Проста схема серіалізації}{simple_schema_dragnit}
        enum Type {
            FLAT = 0;
            ROUND = 1;
            POINTED = 2;
        }

        struct Color {
            byte red;
            byte green;
            byte blue;
            byte alpha;
        }

        message Example {
            uint clientID = 1;
            Type type = 2;
            Color[] colors = 3;
        }
    \end{stdout}

    \subsubsection{Алгоритм роботи} 
    Для початку, визначимо просте повідомленя:

    \begin{snippet}
        message ChangeBitOperation {
            uint bitId = 1;
        }\end{snippet}
    
    Якщо встановити значення, наприклад, в 1.
    Після серіалізації можна побачити наступнку картирну (рисунок \ref{img:drag_algo_1}):

    \printImage[1][0.6]{drag_algo_1}{Приклад роботи алгоритму серіалізації}

    Щоб зрозуміти, як серіалізація працює, спочатку потрібно зрозуміти метод "varint".
    Varint- це метод серіалізації цілих чисел за допомогою одного або декількох байтів.
    Менші числа займають меншу кількість байтів.

    Кожен байт у varint, крім останнього байта, має найзначніший набір бітів (msb) - це вказує на те,
    що надходять ще байти. 
    Нижні 7 біт кожного байта використовуються для зберігання представленого ними комплексу
    представлення числа в групах по 7 біт, найменш значущої групи спочатку.
    
    Строкові рядки кодуються схожим методом, проте для кожного символу використовуеться ключ, ключем є номер поля у повідомлені.

    Наприклад, нехай буде серіалізовано рядок "test" за такою схемою:

    \begin{snippet}
        message StringOperation {
            string str = 1;
        }\end{snippet}
    
    На виході буде отримано значення, яке зображено на рисунку \ref{img:drag_algo_2}:

    \printImage[1][0.6]{drag_algo_2}{Приклад роботи алгоритму серіалізації на строкових даних}

    Як можна побачити, перше значення, 01, і є наш ключ, тобто номер рядка у повідомлені, інші байти - саме слово.
    
    Для більше комплексних данних, завдяки оптимізаціям, можна досягнути значних оптимізацій використання ресурсів.

    Нехай, буде дана більш складна схема:

    \begin{snippet}
        enum Type {
            FLAT = 0;
            ROUND = 1;
            POINTED = 2;
        }

        struct Color {
            byte red;
            byte green;
            byte blue;
            byte alpha;
        }

        message Example {
            uint clientID = 1;
            Type type = 2;
            Color[] colors = 3;
        }\end{snippet}
    
    В даній схемі можна побачити як і користувацькі типи, як struct та enum, так і вкладені данні у повідомлення.
    
    Нехай дана схема будет сереалізована такими данними:

    \begin{snippet}
        {
            "clientID": 100,
            "type": "POINTED",
            "colors": [
                {
                "red": 255,
                "green": 127,
                "blue": 0,
                "alpha": 255
                }
            ]
        }\end{snippet}
    
    Після серіалізації буде отримано такий потік байтів (рисунок \ref{img:drag_algo_3}):

    \printImage[1][0.5]{drag_algo_3}{Приклад роботи алгоритму серіалізації на складних структурах даних}

    Як можна побачити, алгоритм непогано оптимізує вкладені структури данних,
    зберігши по 1 байту на кожне одиничне значення, тобо 5 байтів усього.

    \subsubsection{Реалізація алгоритму}

    Даний алгоритм реалізований мовою Rust, так як вона ідеально підходить для реалізації низькорівневих речей, без великих затрат часу.

    Rust - це системна мультипарадигмена мова програмування, орієнтована на безпеку, особливо безпечну багатопоточність. 
    Rust синтаксично схожа на C ++, але розроблений для забезпечення кращої безпеки пам’яті при збереженні високої продуктивності.

    Для більшої зручності, алгоритм був реалізований в якості модуля бібліотеки, таким чинов він не був кросзалежним від інших модулів,
    його можна легко збирати та тестувати.

    Якщо представити код бібліотеки простою структурною діагармаю, то він буде виглядати так (рисунок \ref{img:dragnit_schema}):

    \printImage[1][1.3]{dragnit_schema}{Структурна схема бібліотеки Dragnit}

    Особливу увагу слід приділити методу skip, а точніше обробці вкладених структур, із трейта Schema (лістинг \ref{code:skip_schema}):

    \begin{code}{Частина коду функції skip, відповідальна за обробку структур}{skip_schema}
        pub fn skip(&self,bb: &mut ByteBuffer,type_id: i32) -> Result<(),()> {
        ...
                let def = &self.defs[type_id as usize];

                match def.kind {
                DefKind::Enum => {
                    if !def.field_value_to_index.contains_key(&bb.read_var_uint()?) {
                    return Err(());
                    }
                },

                DefKind::Struct => {
                    for field in &def.fields {
                    self.skip_field(bb, field)?;
                    }
                },

                DefKind::Message => {
                    loop {
                    let value = bb.read_var_uint()?;
                    if value == 0 {
                        break;
                    }
                    if let Some(index) = def.field_value_to_index.get(&value) {
                        self.skip_field(bb, &def.fields[*index])?;
                    } else {
                        return Err(());
                    }
                    }
                },
            }
        ...
        }\end{code}

    Із цього кода можна побачити, що алгоритм дає змогу обробляти застарілі схеми і навіть пропускати схеми із надлишковою інформацією.

    Такий функціонал є важливим для розподілених систем, оскільки, незважаючи на протокол консенсусу,
    існують субмережі, які мають різні ланцюги розподіленого сховища, проте їм необхідна можливість спілкуватися між собою.

    Наприклад, існує дві компанії, вони використовують один і той же протокол передачі даних, проте мають різні версії протоколу.
    Завдяки серіалізації опційних полів, у них є така можливість, адже алгоритм може без конфліктів викинути поля, які були опціональними
    у більш новій версії протоколу передачі даних.

    \subsection{Порівняння з Protobuf}

    Перевірка буде виконана на заданій схемі:
    
    \begin{snippet}
        message Person {
            string name = 1;
            int id = 2;
            string email = 3;

            enum PhoneType {
                MOBILE = 0;
                HOME = 1;
                WORK = 2;
            }

            message PhoneNumber {
                string number = 1;
                PhoneType type = 2;
            }
        }\end{snippet}

    Тести виконуватимуться для 10000000 операціх серіалізаціЇ та десеріалізації.

    Для серілазації отриманий результат:

    \begin{snippet}
        BenchmarkSerializeToDragnit     10000000  230 ns/op  143 B/op  1 allocs/op
        BenchmarkSerializeToProtobuf    10000000  197 ns/op  80 B/op   1 allocs/op\end{snippet}

    Для десеріалізації отриманий результат:
    
    \begin{snippet}
        BenchmarkSerializeToDragnit     10000000  711 ns/op  421 B/op  15 allocs/op
        BenchmarkSerializeToProtobuf    10000000  461 ns/op  272 B/op   9 allocs/op\end{snippet}

    Графік, отриманий в результаті тестів, зображений на рисунку \ref{img:drag-vs-proto}:

    \printImage[1][0.5]{drag-vs-proto}{Швидкість операцій (зліва - Protobuf, зправа - dragnit)}

    Як можна побачити, в десеріалізації dragnit сильно програє protobuf, проте це пов'язано із перевіркою на опціональні поля,
    яка необхідно для платформи.
    
\subsection{Програмна віртуальна машина}

    Програмна віртуальна машина необхідна для створення захищеної середи виконання криптографічних операцій, та 
    створення агностичної платформи, яка зможе запускатися на будь-якій ОС.

    Так-як потужність сучасного смартфона може позмагатися із простим ПК, то важливо, щоб платформа могла запускатись усюди, 
    при цьому використовуючи мінімум ресурсів.

    У розділі розглянуто реалізація простої стекової машини, парсера програмної мови, оптимізація її швидкодії та використання ресурсів, 
    та реалізація базових криптографічних алгоритмів на створеній для віртуальної машини мові. 

    \subsubsection{Архітектура}

    Обираючи із стекової та реєстрової машини, було обрану стекову, через її простоту, проте і значну швидкодію, ніж реєстрова.

    Однією з важливих причин розроблення мов на основі стека є те,
    що мінімалізм їх семантики дозволяє просту інтерпретацію та реалізацію компілятора, а також оптимізацію.

    Отже, однією з практичних переваг такої парадигми є те,
    що вона дозволяє розробникам легко будувати над ними складніші речі та парадигми.

    Серед переваг також можна назвати:

    \listDefault{
        \item Час процесора - вартість часу на розподіл пам'яті в стеку практично безкоштовна:
        неважливо, виділяється одна чи тисячуа цілих чисел, все, що потрібно, - це операція зменшення покажчика стека;

        \item Витік пам'яті - під час використання стеку немає витоків пам'яті. Це відбувається природно без додаткових накладних витрат
        на вирішення цього питання. Пам'ять, яку використовує функція,
        повністю звільняється при поверненні з кожної функції навіть при
        обробці винятків або використанні longjmp (відсутність посилань на підрахунок, збирання сміття, тощо);
        \item Фрагментація - стеки також уникають фрагментації пам'яті природним шляхом.
        Можна домогтися нульової фрагментації без будь-якого додаткового коду для вирішення цього питання,
        наприклад, пулу об'єктів або розподілу пам'яті на платформі.
        \item Локальність - дані в стеку надають перевагу локальному збереженню, використовуючи кеш-пам'ять та уникаючи змін сторінок.
    }

    Віртуальна машина реалізована мовою Rust.

    Архітектура машини досить проста. За значення стеку відповідає StackValue, зображений на рисунку \ref{img:vm_algo_1}.

    \printImage[1][0.4]{vm_algo_1}{Структура StackValue}

    Основою ж є трейт Machine, який дозволяє виконувати операції \ref{img:vm_algo_2}:

    \printImage[1][0.5]{vm_algo_2}{Структура VM}

    Більш детально реалізація буде розглянута у наступних розділах.

    Список операцій, які підтримує VM:
    \printTableWithCaption{
        \makeLongTable{
            Ім'я операції & Opcode & Операція
        }{
            LShift      &   <<  & Push(Num(a << b)) \\
            RShift      &   >> & Push(Num(a >> b))\\
            Plus        &   +  &  Push(Num(a + b)) \\
            Minus       &   -  & Push(Num(b - a)) \\
            Multiply    &   *  &  Push(Num(a * b)) \\
            Divide      &   /  & Push(Num(b / a)) \\
            ToInt       & cast\_int & Push(Num(a.parse::)) \\
            ToStr       & cast\_str & Push(String(format!(a)))\\
            Println& println&Println(a)\\
            Equals&==&Push(Bool(a == b))\\
            Or&or&Push(Bool(a || b))\\
            BitOr&|&Push(Num(a | b))\\
            And&and &Push(Bool(a \&\& b))\\
            Not&not&Push(Bool(!a))\\
            Xor&xor&Push(Num(a \^{} b))\\
            LessThan&<&Push(Bool(b < a))\\
            LessThanOrEqualTo&<=& Push(Bool(b <= a))\\
            GreaterThan&>&Push(Bool(b > a))\\
            GreaterThanOrEqualto&>=&Push(Bool(b >= a))\\
            Mod&\%&Push(Num(b \% a))\\
            If&if&Push(if cond { t } else { f })\\
            Jump&jmp&Jump(a as usize)\\
            Duplicate&dup&PushTwo(val.clone(), val)\\
            Drop&drop&NA\\
            Rotate&rot&PushThree(b, a, c)\\
            LongRot&lrot&PushThree(b, c, a)\\
            ShortRot&srot&PushThree(a, c, b)\\
            Fup&fup&PushFour(d,b,c,a)\\
            Swap&swap&PushTwo(a, b)\\
            SleepMS&sleep\_ms&Sleep(a as u64)\\
            Exit&exit&Stop(exit\_code as i32)\\
            Hex&hex&Push(String(format!(a)))\\
            Stop&stop&Stop(0)\\
            Read&read&ReadLn\\
            Over&over&PushThree(b.clone(), a, b)\\
            Call&call&Call(a as usize)\\
            Return&return&Return\\
        }{|p{5cm}|p{5cm}|p{6cm}|}
    }{
        Таблиця підримуємих операцій
    }{vm-operations}
    
    \subsubsection{Огляд реалізації та роботи VM}
        Для початку, потрібно розглягути одну з найважливіших функцій - токенізатор (лістинг \ref{code:vm_tokenizer}):

        \begin{code}{Частина функції обробки токенів}{vm_tokenizer}
            ...
            for c in input.chars() {
                if state.ignore_til_eol {
                    if c == '\n' {
                        state.ignore_til_eol = false;
                    }
                    continue;
                }

                match c {
                    '"' => {
                        state.push_char(c);
                        if !state.prev_is_escape && state.token.len() > 1 {
                            state.push_token()?;
                        } else {
                            state.prev_is_escape = false;
                        }
                    }
                    '\\' => {
                        if state.prev_is_escape {
                            state.prev_is_escape = false;
                            state.push_char(c);
                        } else {
                            state.prev_is_escape = true;
                        }
                    }
                    '#' => {
                        if !state.token_is_string() {
                            state.ignore_til_eol = true;
                        } else {
                            state.push_char(c);
                        }
                    }
                    ' ' | '\n' | '\t' | '\r' => {
                        if state.token_is_string() {
                            state.push_char(c);
                        } else {
                            state.push_token()?;
                        }
                    }
                    _ => {
                        state.push_char(c);
                    }
                }
            }
            ...\end{code}

    Як можна побачити, токенізатор та парсер реалізовані не через регулярні вирази, так як це було би ресурсовитратно.
    Натомість, токенізатор реалізований через PEG граматику.

    Граматики розбиття (PEG) - це досить молоде відкриття у світі граматики та розбору.
    Вони були запропоновані Брайаном Фордом у 2004 році.

    Здебільшого вони описують спосіб зчитування входів та
    деструкцію їх на правила замість пояснення,
    як створювати рядки, як це роблять інші граматики.
    Брайан Форд створив їх, маючи на увазі мови програмування,
    і як такі вони добре підходять для опису комп'ютерних/машинних граматик (DSL, JSON, граматика D, навіть самої граматики PEG).

    Принципова відмінність між безконтекстними граматиками та граматиками вираження розбору
    полягає в тому, що оператор вибору PEG впорядкований.
    Якщо перша альтернатива успішна, друга альтернатива ігнорується.
    Таким чином, упорядкований вибір не є комутативним,
    на відміну від невпорядкованого вибору, як у граматиках без контексту.
    Впорядкований вибір аналогічний операторам програмного забезпечення, який доступний у деяких мовах програмування логіки.

    Наслідком цього є те, що якщо CFG транслітерується безпосередньо до
    PEG, будь-яка двозначність у першому вирішується шляхом детермінованого вибору одного
    дерева розбору з можливих синтаксичних аналізів.
    Ретельно вибираючи порядок, у якому вказані альтернативи
    граматики, програміст має великий контроль над тим, яке дерево аналізу буде вибрано.

    Як булеві контекстні граматики без розбору, граматики вираження розбору
    також додають синтаксичні предикати. Оскільки вони можуть використовувати довільно складний підвираз,
    щоб "дивитись вперед" у вхідний рядок, не фактично використовуючи його,
    вони забезпечують потужну синтаксичну функцію пошуку та розбірливості,
    зокрема при упорядкуванні альтернатив не можуть вказати потрібне дерево розбору.


    На лістингу \ref{code:simple_vm_grammar} зображено простий алгоритм, написаний на мові віртуальної машини,
    який дозволяє просумувати числа із заданим інтервалом.
    
    \begin{code}{Приклад вихідного коду VM}{simple_vm_grammar}        
        loop:
        over + dup println
        rot over over
        >= finish continue if jmp
        continue:
        rot rot
        loop jmp
        finish:
        stop
        get:
        println read cast_int return\end{code}

    Даний алгоритм простий і розглядатись його виконання не буде, він наведений лише для прикладу граматики.
    
    Також, однією із особливостей є предкомпіляція коду, на основі PEG граматики можна заздалегіть розібрати, слід виконувати код, чи ні.

    Частина препроцесура, зображена на лістингу \ref{code:vm_preprocess}, необхідна, щоб заздалегіть визначати мітки і перетворювати
    їх в оптимізований код, якщо препроцесор може його оптимізувати, або навіть запобігати небажаних патернів.

    \begin{code}{Частина функції VM, відповідаюча за передбачення}{vm_preprocess}        
        let replacements = {
            let mut labels_meta: HashMap<&str, (Vec<usize>, Vec<usize>)> = HashMap::new();
            let mut replacements = vec![];

            for (idx, value) in code.iter().enumerate() {
                if let StackValue::Label(ref s) = *value {
                    let entry = labels_meta.entry(s).or_insert((vec![], vec![]));
                    entry.0.push(idx + 1);
                } else if let StackValue::PossibleLabel(ref s) = *value {
                    let entry = labels_meta.entry(s).or_insert((vec![], vec![]));
                    entry.1.push(idx);
                }
            }

            for (key, val) in labels_meta {
                if val.0.len() > 1 {
                    return Err(StackError::MultipleLabelDefinitions {
                        label: (*key).into(),
                        locations: val.0.clone(),
                    });
                } else if val.0.is_empty() && !val.1.is_empty() {
                    return Err(StackError::UndefinedLabel {
                        label: (*key).into(),
                        times: val.1.len(),
                    });
                } else {
                    replacements.push((val.0[0], val.1));
                }
            }
            replacements
        };\end{code}
    
    
    Для перевірки працездатності, для початку, реалізована проста шифр XOR, вихідний код наведей на лістингу \ref{code:vm_xor}.

    \begin{code}{Код VM для шифру XOR}{vm_xor}
        loop:
        lrot
        dup
        lrot
        swap
        xor
        println
        swap
        1 - 
        dup
        0
        <= finish continue if jmp
    continue:
        swap
        loop jmp
    finish:
        stop\end{code}
    
    Функція приймає на вхід набір байтів (які можуть представляти текст у будь-якому кодуванні), довжину слова, та ключ, яким буде
    шифруватися повідомлення.
    
    На виході буде отримано перевернутий масив байтів. Первернутий, тому що повна ротація стека дуже накладна операція, нативні функції
    зможуть перевернути слово швидше.

    Для відладки коду можна скористатися дебагером, який вбудовано у VM, проте користувацький інтерфейс, для простоти і швидкості
    реалізований у якості консольного додатку.

    Для перевірки, буде зашифровано слово "test" (74 65 73 74) за ключем 10.

    На рисунку \ref{img:xor_stack} зображена відладка стеку під час виконання:

    \printImage[1][0.5]{xor_stack}{Відладка виконання XOR шифра}
    
    На рисунку \ref{img:xor_res} зображений результат виконання операції:

    \printImage[1][0.2]{xor_res}{Результат виконання XOR шифра}
    
    Якщо порівняти із будь-якою іншою реалізацію (рисунок \ref{img:xor_web_res}), то можна побачити, що алгоритм працює правильно.

    \printImage[1][0.5]{xor_web_res}{Перевірка правильності виконання алгоритму}

    На рисунку \ref{img:xor_flame} зображений результат продуктивності виконання XOR коду на VM:

    \printImage[1][0.5]{xor_flame}{Діаграма процесів VM при виконанні XOR алгоритму}

    Найбільше часу займають функції dispatch та run, це очевидно, так як вони являються основними функціями VM.
    Проте, дуже часто в алгоритмі використовується метод 'dup', це не погано,
    проте, можна подумати, чи є ще можливість оптимізувати алгоритм

    \subsubsection{Реалізація та оптимізація алгоритму Adler32}
    
    Adler-32 - алгоритм контрольної суми, який був винайдений Марком Адлером у 1995 році і є модифікацією контрольної суми Флетчера.
    Порівняно з CRC, він більш швидкий. Adler-32 надійніший за Fletcher-16 і трохи менш надійний, ніж Fletcher-32.

    Adler-32 використовується в бібліотеці компресії даних zlib, яка славиться своєю швидкістю, а швидкість є дуже цінним показником
    в децентралізованих мережах.

    Чим швидше будуть перевірятися пакети на валідність, тим швидше буде розрішуватися консенсус у мережі.

    Вихідний код алгоритму Adler32 представлений на лістингу \ref{code:adler-source}.

    \begin{code}{Вихідний код функції Adler32}{adler-source}
        loop:
        fup
        +
        65521
        swap
        %
        dup
        lrot
        +
        65521
        swap
        %
        swap
        rot
        1
        -
        dup
        0
        <= finish continue if jmp
    continue:
        srot
        loop jmp
    finish:
        srot
        swap
        16
        <<
        |
        hex
        println
        stop\end{code}

    Код стекової машини трохи складний для людини, якщо його інтерпретувати у мову С, як представлено на лістингу \ref{code:adler-c}.

    \begin{code}{Вихідний код функції Adler32 на мові C}{adler-c}
        uint32_t adler32(const void *buf, size_t buflength) {
            const uint8_t *buffer = (const uint8_t*)buf;

            uint32_t s1 = 1;
            uint32_t s2 = 0;

            for (size_t n = 0; n < buflength; n++) {
                s1 = (s1 + buffer[n]) % 65521;
                s2 = (s2 + s1) % 65521;
            }     
            return (s2 << 16) | s1;
        }\end{code}

    Функція приймає на вхід набір байтів (які можуть представляти текст у будь-якому кодуванні) та довжину слова.
    
    На виході буде отримано значеня хеш у 16-ічній системі. Для простоти відладки віртуальна машина
    може працювати із перетвореням даних у різні формати, проте для більшої оптимізації такі функції необхідно реалізовувати
    через функціонал самої VM, так як це буде не тільки швидше, а й безпечніше.

    Для перевірки буде дана проста послідовність байтів - (1, 2, 3), так як такий результат не важко порахувати,
    хешем послідовності буде значення 0x11.

    На рисунку \ref{img:adler_stack} зображена відладка стеку під час виконання:

    \printImage[1][0.5]{adler_stack}{Відладка виконання Adler-32}
    
    На рисунку \ref{img:adler_res} зображений результат виконання операції:

    \printImage[1][0.2]{adler_res}{Результат виконання Adler}

    На рисунку \ref{img:adler_flame} зображений результат продуктивності виконання Adler коду на VM:

    \printImage[1][0.7]{adler_flame}{Діаграма процесів VM при виконанні Adler алгоритму}
    
    \subsubsection{Тестування продуктивності}

    Для тестування продуктивності реалізованих алгоритмів, буду взято 1000, 10000 та 100000 довільних наборів даних.

    Для функції XOR, отриманий результат:
    
    \begin{snippet}
        BenchmarkXOR     1000   180 ns/op  23 B/op  1 allocs/op
        BenchmarkXOR     10000  201 ns/op  43 B/op  2 allocs/op
        BenchmarkXOR     100000 215 ns/op  51 B/op  5 allocs/op\end{snippet}

    Для функції Adler32, отриманий результат:

    \begin{snippet}
        BenchmarkAdler32     1000   223 ns/op  34 B/op  1 allocs/op
        BenchmarkAdler32     10000  290 ns/op  67 B/op  1 allocs/op
        BenchmarkAdler32     100000 415 ns/op  78 B/op  15 allocs/op\end{snippet}
    
    Хоча Adler32 показує просідання по швидкодії, при великих наборах даних, проте це можна уникнути шляхом оптимізації функцій.
    В цілому, реалізації алгоритмів показують досить непоганий результат.

\subsection{Механізм обробки подій}

    Для оптимального використання системних ресурсів, необхідний архітектурний підхід, який сможе ефективно управлять ресурсами,
    при цьому не даючи системі можливості простоювати, використовуючи ресурси раціонально.

    Для високонавантаженої децентралізованої платформи підійде архітектурний патерн Reactor.

    Патер Reactor є однією з реалізації архітектури, керованої подіями.
    Простіше кажучи, він використовує єдину петлю подій, що блокує події,
    що виділяють ресурси, і передає їх відповідним обробникам та зворотним викликам.

    Не потрібно заблоковувати I/O канали, доки обробники та зворотні виклики для подій реєструються для їх догляду.
    Події стосуються таких випадків, як нове вхідне з'єднання, готове до читання, готове до запису тощо.
    Ці обробники/зворотні виклики можуть використовувати пул потоків у багатоядерних середовищах.

    Є два важливих компонента в архітектурі Reactor Pattern:

    \listEnumerated {
        \item Reactor - реактор працює в окремій потоці, і його завдання полягає в тому,
        щоб реагувати на події вводу-виводу, направляючи роботу до відповідного обробника.
        Це як телефонний оператор у компанії, який відповідає на дзвінки від клієнтів і передає лінію відповідному контакту;
        \item Handler - обробник виконує фактичну роботу, яка повинна бути виконана з подією вводу-виводу,
        подібно до фактичного співробітника компанії, з яким клієнт хоче поговорити.
        Реактор реагує на події вводу-виводу шляхом відправлення відповідного обробника. Обробники виконують неблокуючі дії.
    }

    Архітектурна схема Reactor дозволяє керувати подіями додатка для демультиплексування та диспетчеризації запитів на послуги,
    які доставляються до програми від одного або декількох клієнтів.

    Один реактор буде продовжувати шукати події та інформуватиме відповідного обробника подій,
    щоб обробляти його, як тільки подія запускається.

    Шаблон реактора - це схема дизайну для синхронного демультиплексування та порядку подій у міру їх
    надходження.

    Він отримує повідомлення, запити та з'єднання, що надходять від декількох одночасно клієнтів, і обробляє ці
    повідомлення послідовно, використовуючи обробники подій. Мета схеми дизайну Reactor -
    уникнути поширеної проблеми створення потоку для кожного повідомлення, запиту та з'єднання.
    Потім він отримує події від набору обробників і поширює їх послідовно до відповідних обробників подій.

    Для реалізації патерну, за основу був обраний файловий дескриптор Unix, проте для підтриманя агностичності платформи, також необхідно
    реалізовувати механізми для Windows систем та FreeBSD систем, так вони реалізують інші механізми.

    \subsubsection{Реалізація через Epoll}

    Так як патерн буде реалізовуватся на мові Rust, необхідно отримати доступ до нативних функцій epoll.

    На лістингу \ref{code:epoll_wrapper} зображено обгортку над нативними функціями, яка дозволяє робити безпечні виклики та обробляти помилки.
    
    \begin{code}{Обгортка над нативними функціями epoll}{epoll_wrapper}
    use libc::c_int;
    bitflags! {

        #[repr(C)]
        pub flags EventType: u32 {
            const EPOLLIN = 0x001,
            const EPOLLOUT = 0x004,

            const EPOLLPRI = 0x002,

            const EPOLLERR = 0x008,

            const EPOLLHUP = 0x010,
            const EPOLLRDHUP = 0x2000,

            const EPOLLEXCLUSIVE = 1 << 28,

            const EPOLLWAKEUP = 1 << 29,
            const EPOLLONESHOT = 1 << 30,

            const EPOLLET = 1 << 31,

            const EPOLLRDNORM = 0x040,

            const EPOLLRDBAND = 0x080,

            const EPOLLWRNORM = 0x100,

            const EPOLLWRBAND = 0x200,

            const EPOLLMSG = 0x400,
        }
    }   

    #[derive(Clone, Copy, Debug)]
    #[repr(C,packed)]
    pub struct Event {
        pub events: EventType,
        pub data: u64
    }

    impl Default for Event {
        fn default() -> Event {
            Event { events: EPOLLIN, data: 0 }
        }
    }

    extern {
    pub fn epoll_create(size: c_int) -> c_int;

    pub fn epoll_create1(flags: c_int) -> c_int;

    pub fn epoll_ctl(epfd: c_int,
                    op: c_int,
                    fd: c_int,
                    event: *mut Event) -> c_int;

    pub fn epoll_wait(epfd: c_int,
                    events: *mut Event,
                    maxevents: c_int,
                    timeout: c_int) -> c_int;
    }\end{code}

    Такий підхід дозволить безпечно користуватися системними викликами ядра Linux.

    Отже, все що необхідно, це реалізувати простий алгоритм, наведений на рисунку \ref{img:reactor-algo}.

    \printImage[0.5][0.6]{reactor-algo}{Схема роботи патерна Reactor}

    Для тестів, в якості ресурсів, використовуються сокети, які обробляють прості HTTP запити.

    Для заміру ефективності, використаємо утиліту wrk:

    \begin{snippet}
        wrk -c100 -d1m -t8 http://127.0.0.1:30000 -H "Host: 127.0.0.1:3000" -H "Accept-Language: en-US,en;q=0.5" -H "Connection: keep-alive"
        Running 1m test @ http://127.0.0.1:3000
        8 threads and 100 connections
        Thread Stats   Avg      Stdev     Max   +/- Stdev
            Latency   523.52us   86.21us  18.64ms   92.15%
            Req/Sec    21.1k     1.95k   32.57k    72.54%
        10543532 requests in 1.00m, 1.40GB read
        Requests/sec: 175725.53
        Transfer/sec:     23.3MB
    \end{snippet}

    Із замірів можна побачити, що патер справді ефективно використовує ресурси, не даючи системі простоювати.
    Також, патерн можна доповнити, запустивши його на мультиядерній системі, цим самим значно покращити продуктивність.

    \subsubsection{Event Loop}
    
    Для того, щоб покращити результати чистої обгортки над epoll,
    слід також реалізувати сховище для подій, де будуть зберігатися усі ще не викликані події. 
    Саме для цього і необхідно реалізувати Event Loop.

    Структурно, Event Loop виглядатиме, як зображено на рисунку \ref{img:event-loop-struct}.

    \printImage[1][0.4]{event-loop-struct}{Структурна схема Event Loop}

    Event Loop матиме свій вектор, до якого epoll, після спрацювання, буде додавати доступні десриптори.
    За допомогою EventLoopIterator, реактор буде перебирати ці події та обробляти.

    Для заміру ефективності, використаємо утиліту wrk:

    \begin{snippet}
        wrk -c100 -d1m -t8 http://127.0.0.1:30000 -H "Host: 127.0.0.1:3000" -H "Accept-Language: en-US,en;q=0.5" -H "Connection: keep-alive"
        Running 1m test @ http://127.0.0.1:3000
        8 threads and 100 connections
        Thread Stats   Avg      Stdev     Max   +/- Stdev
            Latency   500.11us   84.11us  18.34ms   90.47%
            Req/Sec    20.5k     1.77k   31.21k    73.21%
        10765475 requests in 1.00m, 1.45GB read
        Requests/sec:  179424.58
        Transfer/sec:     24.16MB
    \end{snippet}

    Як можна побачити, деякі параметри трохи покращились, наприклад реактор став обробляти більше запитів, тому оптимізацію можна вважати вдалою.

\subsection{Протокол консенсусу (HBBFT)}

    Дивовижний успіх криптовалют призвів до сплеску інтересів у розгортанні широкомасштабних,
    високоміцних, візантійських протоколів (BFT) для критично важливих програм,
    таких як фінансові транзакції.
    
    Хоча загальноприйнята мудрість полягає у побудові синхронного протоколу, такого як PBFT (або його варіація),
    такі протоколи критично покладаються на припущення про мережевий таймінг і лише гарантують життєздатність,
    коли мережа веде себе як очікувалося. Очікується, що ці протоколи не підходять для цього сценарію розгортання.
    
    Тому, було за основу протокола консенсуса було обрано HoneyBadgerBFT,
    перший практичний синхронний протокол BFT, який гарантує життєздатність,
    не роблячи жодних припущень про терміни. Рішення базується на новому протоколі atomicbroadcast,
    який досягає оптимальної асимптотичної ефективності. Така система може досягти пропускної здатності десятків тисяч транзакцій за секунду,
    і масштабує понад сотню вузлів у широкій мережі.

    У HoneyBadgerBFT вузли отримують транзакції як вхідні дані і зберігають їх у своїх
    (необмежених) буферах.
    Протокол продовжується в "епохи", де після кожної епохи до заповненого журналу додається нова партія транзакцій.
    На початку кожної епохи, вузли вибирають підмножину транзакцій у своєму буфері
    та надають їх як вхід до екземпляра протоколу рандомізованої угоди.
    Наприкінці домовленості про протокол вибирається остаточний набір транзакцій для цієї епохи.
    На цьому високому рівні такий підхід схожий на існуючі асинхронно-атомні протоколи атомного мовлення,
    зокрема на Cachinet, основи для широкомасштабної системи обробки транзакцій (SINTRA).
    Примітив дозволяє кожному вузлу запропонувати значення і гарантує, що кожен вузол видає загальний вектор,
    що містить вхідні значення принаймні $N-2$ правильних вузлів.
    Історично, щоб побудувати атомну трансляцію з цього примітиву - кожен вузол просто пропонує підмножину транзакцій
    з передньої своєї черги і виводить об'єднання елементів у узгоджений вектор. Однак є дві важливі проблеми.

    \listEnumerated{
        \item Досягнення стійкості до цензури - вартість примітивів залежить безпосередньо від розміру наборів транзакцій,
        запропонованих кожним вузлом. Оскільки вектор виводу містить щонайменше N множини, ми можемо, таким чином,
        підвищити загальну ефективність, гарантуючи, що вузли пропонують переважно нероз'єднані набори транзакцій,
        здійснюючи тим самим більше чітких транзакцій за одну партію за однакові витрати.
        Отже, замість простого вибору першого елемента s з його буфера, кожен вузол протоколу пропонує вибірковий вибір,
        таким чином, що кожна транзакція в середньому пропонується лише одним вузлом;
        \item Практична пропускна здатність - хоча теоретична можливість асинхронних примітивів та атомного мовлення відома,
        їх практична ефективність не є відомою. Тому цікавим питанням є те, чи можуть такі протоколи досягати високої пропускної здатності.
    }

    Honey Badger BFT також забезпечує остаточність блоку та ефективно справляється із непередбачуваною поведінкою.
    Крім того, він пропонує кілька ключових переваг перед pBFT, в результаті чого процес консенсусу є ефективним
    та стійким до атаки. Ці покращення включають:

    \listDefault{
        \item Швидкість блоку не потрібно налаштовувати на основі сценаріїв або припущень, вона відповідає справжній швидкості мережі.
        \item Консенсус без лідерів. На відміну від pBFT, механізм консенсусу HBBFT не вимагає,
        щоб лідерський вузол пропонував транзакції. Кожен вузол є пропонувачем.
        Це виключає потенційні атаки, коли лідерний вузол може зупинятися на невизначений термін,
        приводячи до зупинки всю мережу.
        \item Ефективне, безпечне розповсюдження повідомлень.
        На основі алгоритмів, розроблених Міллером та ін., ,Honey Badger використовує кілька методів для
        ефективного шифрування, розбиття та надсилання повідомлень невеликими шматками.
        Це економить пропускну здатність і створює надзвичайно ефективний процес.
    }
    \subsubsection{Архітектура модуля}

    Процес роботи алгоритму зображений на рисунку \ref{img:bft-flow}

    \printImage[1][0.6]{bft-flow}{Схема роботи алгоритма HBBFT}

    Так як алгоритм асинхроний, це дає ще більшу оптимізацію використання ресурсів і менший час простою.

    Повна модель BFT складається з 4 основних частин:

    \listDefault{
        \item Модуль консенсусу, модуль алгоритму консенсусу включає перевірку підписів, генерацію доказів, перевірку версій тощо;
        \item Машина стану, машина стану BFT орієнтована на консенсусну пропозицію;
        \item Транспортний модуль, мережа для модуля консенсусу для зв'язку з іншими модулями;
        \item Wal-модуль, місце збереження журналів BFT.
    }
    

    Метою HBBFT є повернення узгоджених партій транзакцій.
    Він не завершує остаточну обробку блоку (наприклад, оновлення залишків рахунків, виконання смарт-контрактів та підписання блоків)
    і не додає блоки до блокчейн. Це відповідальність програми, яку HBBFT використовує для спілкування з блок-ланцюгом (наприклад, Parity).

    HBBFT розділений на модулі, і кожен модуль обробляє окрему частину процесу консенсусу.

    Зв'язок між модулями, як правило, у форматі "чорної коробки".
    Це означає, що модуль просто отримує вхід і забезпечує вихід.
    Модулю не потрібно знати, як працює інший модуль, лише який вихід очікувати та вхідний показник.
    Таке розділення проблем дозволяє легко оптимізувати та підтримувати бібліотеку.

    Для того, щоб отримати повну картину про те, як працює HBBFT, необхідно змоделювати ситуацію.

    У наступному сценарії, змоделюємо ситуацію, коли один користувач передає 1 токен іншому користувачеві:
    \listDefault{
        \item Користувач увійдете у систему і використовуєте інтерфейс,
        щоб заповнити суму для надсилання та адресу гаманця іншого користувача;
        \item Вузол блокчейна отримує цей запит на транзакцію через Інтернет-запит;
        \item Вузол вводить цю транзакцію в HBBFT, ставлячи її в чергу (QHB). QHB розміщує цю транзакцію разом з іншими отриманими у черзі транзакцій,
        що очікують на розгляд;
        \item Починається нова епоха. Випадковий процес визначає, які транзакції слід включити в наступний блок;
        \item QHB готує список. У мережі є 21 валідатори, тому розмір списку становить 1/21 розмір блоку;
        \item Перелік транзакцій подається до Honey Badger. HB шифрує список за допомогою порогової криптографії,
        створюючи скриптовану версію, яка містить транзакцію, але її неможливо прочитати;
        \item Вклад подається до алгоритму підмножини;
        \item Підмножина передає його в надійний мовний екземпляр, позначений ідентифікатором вузла (наприклад, Вузол 1).
        Це розподіляє внесок на кожен інший вузол мережі;
        \item Як тільки кожен вузол отримав зашифрований внесок, вони знають, що всі інші правильні вузли
        також отримають цей внесок. Вони голосують "Y" у екземплярі угоди з іменем "Вузол 1";
        \item Кожен вузол повертає Y для екземпляра договору, позначеного Вузол 1, тобто всі вони згодні з тим,
        що цей внесок повинен бути включений як частина наступного блоку;
        \item Домовленість досягнута. Внесок повертається назад в стек;
        \item Підмножина повертає внески HB 21 на 21 вузол, всі зашифровані;
        \item В НВ вся мережа співпрацює, щоб розшифрувати внесок;
        Кожен вузол отримує 21 список транзакцій, які розшифровуються. Ці списки включають оригінальну транзакцію;
        \item Потім QHB робить об'єднання внесків і створює єдиний остаточний перелік транзакцій, про який узгодили всі вузли;
        \item Цей остаточний список надсилається з QHB та повертається до клієнта програми;
        \item Блокчейн виконує транзакцію і оголошує її як частину наступного блоку.
        Токен передається іншому користувачеві, в блокчейні з'являється запис.
    }
    
    \subsubsection{Симуляція та тестування мережі}

    Оскільки на практиці важко сказати, наскільки алгоритм буде ефективним в тій чи іншій ситуації, то 
    необхідно змоделювати різні умови.

    На рисунку \ref{img:badg_sim_1} зображений результат симуляції для 10 вузлів.

    \printImage[1][0.6]{badg_sim_1}{ Симуляція HBBFT мережі із 10 вузлів}
    
    Такий варіант є, практично, ідаельним, тому ми бачимо гарний результат.

    Теперь спробуємо змоделювати мережу, де деякі вузли можуть відмовити (рисунок \ref{img:badg_sim_2}):

    \printImage[1][0.6]{badg_sim_2}{ Симуляція HBBFT мережі із 15 вузлів, де 2 вузли несправні}

    Алгоритм показав дуже гарний результат, навіть при наявності несправних вузлів, що простий BFT ніяк не може гарантувати.

    Інша модель покаже, наскільки алгоритм справляється із підвищеням трафіку в мережі (рисунок \ref{img:badg_sim_3}):

    \printImage[1][0.6]{badg_sim_3}{ Симуляція HBBFT мережі із 10 вузлів, 500 пакетів кожної ери}

    Із результатів можна побачити, що алгоритм справився усьго за 3 епохи, причому в дуже швидкому темпі.
    Це значить, що алгоритм масштабований до високого трафіку и цілком може справлятися з великими нагрузками.

\subsection{Протокол GOSSIP}  

    Так як мережа децентралізована і постійно розширювана, то без discovery алгоритму не обійтись.

    Переваги GOSSIP:
    \listDefault{
        \item Масштабованість - оскільки загалом потрібна складність O(logN), щоб досягти всіх вузлів,
        де N - кількість вузлів.
        Також кожен вузол надсилає лише фіксовану кількість повідомлень,
        незалежних від кількості вузлів у мережі. Вузол не чекає підтверджень,
        і він не вживає жодних дій відновлення, якщо підтвердження не надійде. 
        Система може легко масштабувати мільйони процесів;
        \item Відмовостійкість - має можливість працювати в мережах з неправильним і невідомим зв’язком.
        У цих ситуаціях вони справді добре працюють, тому що, як ми збираємось,
        вузол ділиться однією і тією ж інформацією кілька разів на різні вузли,
        тому, якщо вузол не доступний, інформація поділяється все одно через інший вузол.
        Іншими словами, існує багато маршрутів, за якими інформація може надходити від її
        джерела до місця призначення;
        \item Міцний - жоден вузол не відіграє певної ролі в мережі, тому несправний вузол не завадить
        іншим вузлам продовжувати надсилати повідомлення;
        \item Кожен вузол може приєднатися або вийти, коли йому заманеться,
        не порушивши загальну якість обслуговування системи. Вони не є надійними за будь-яких обставин;
        \item Конвергентна консистенція - протоколи пліток досягають експоненціально швидкого поширення
        інформації і, отже, швидко переходять в експоненціально стан до глобально послідовного
        стану після настання нової події, за відсутності додаткових подій, 
        поширючи будь-яку нову інформацію по всіх вузлах, на які буде впливати
        інформація протягом логарифмічного розміру системи;
        \item Надзвичайно децентралізована - плітки пропонують надзвичайно децентралізовану форму
        пошуку інформації, і її затримки часто прийнятні,
        якщо інформація насправді не буде використана негайно.
        \item Простота реалізації (рисунок \ref{img:gossip_pseudo})
    }

    \printImage[0.8][0.6]{gossip_pseudo}{ Псевдокод простого алгоритму GOSSIP }

    \subsubsection{Моделювання мережі з протоколом GOSSIP}

    Для початку, модель буде складатися із 20 вузлів, щоб переконатися, що алгоритм спроможний відкрити всі шляхи.

    На рисунку \ref{img:gossip_sim_1} зображений стан кожного з вузлів.

    \printImage[1][0.6]{gossip_sim_1}{ Симуляція GOSSIP протокола на 20 вузлах}

    На рисунку \ref{img:gossip_sim_2} зображений коефіцієнти кластерів кожного з вузлів.

    \printImage[1][0.6]{gossip_sim_2}{ Коефіцієнти кластерів для вузлів }

    Як і очікувалось, деякі коефіцієнти рівні 0, так як мержа дуже розріджена, а кількість сусідів у вузлів 
    не дуже велика.

    Такий підход дозволяє побудувати насправді росподілений підход до віявленя вузлів у мережі (рисунок \ref{img:gossip_sim_3})

    \printImage[1][0.6]{gossip_sim_2}{ Граф зв'язку вузлів у мережі }

\subsection{Реалізації тестових додатків}

В даному розділі приведені приклади тестових додатків, створених на платформі.

\subsubsection{Груповий чат}

\subsection{Висновки}

\startSection{РОЗРОБЛЕННЯ СТАРТАП-ПРОЕКТУ }
\startStartupSection
Описані в магістерській роботі підходи до проектування платформи для створення децентралізованих додатків
можна застосовувати для розробки спеціалізованих платформ та модифікації існуючих.
За ідею для стартап-проекту була вибрана технологія масштабованої децентралізованої платформи з можливістю подальшої її модифікації.


\printTableWithCaption{
    \makeTable{
        Зміст ідеї & Напрямки застосування & Вигоди для користувача
    }{
        Децентразівона масштабована платформа для створення блокчейн додатків&
        1. Розробка децентралізованих додатків&
        1. Можливість розгортати додатки у децентралізованій мережі \\
        & & 2. Зменшення витрат ресурсів для створення додатків\\
    }{|p{5cm}|p{8cm}|p{5cm}|}
}{
    Опис ідеї стартап-проекту
}{startup-idea-desc}

\startStartupSection
\startStartupSection
\startStartupSection
\startStartupSection


\startSection{ВИСНОВКИ}
\startSection{СПИСОК ВИКОРИСТАНИХ ДЖЕРЕЛ}

\end{document}