\documentclass{lib/styles/default-style}

\begin{document}
Магістерська дисертація на здобуття ступеня магістру на тему “Моделі і  методи  швидкого  створення  веб-застосувань”: 125с., 33рис.,  22  табл., 2 додатки, 26 джерел.

У магістерській  дисертаціїрозглядається  проблема  використання формальних засобів для швидкого створення ефективних web-додатків одного класу. Пропонується концептуальний підхід до її вирішення на основі аналізу особливостей  побудови  web-додатків.  Підхід  базується  на  визначенні стандартної архітектури web-додатків і виборі її складових з використанням формальних методів відповідно до вимог користувача. Наводитьсяформальна логічна  система  з  використанням  якої  проектування  web-додатків здійснюється як процес виведення заданої відповідно до потреб користувача формули,  який  визначає  схеми  виконання  модулів  системи. 

Важливою особливістю  підходу  є  можливість  3D-візуалізації  процесу  проектування системи, що створює умови для ефективної взаємодії розробників і машинних інструментів розробки.

\end{document}