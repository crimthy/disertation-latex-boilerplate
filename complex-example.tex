\documentclass{lib/styles/default-style}
\usepackage{lib/docs/title}
\usepackage{lib/docs/task}

\begin{document}
\pagestyle{default-numbered}
\startPageCountFrom{3}

\tableofcontents

\newpage

\unnumberedSection{НАЙКРАЩИЙ ВСТУП}

Створення  інформаційних  систем  сьогодні  здійснюється  
на  основі сучасних  методологічних  концепцій,
які  успадкували  найважливіші  ідеї класичних методологій типу
SADT,IDEFO, одночасно збагативши їх новими ідеями.

\startSection{ВСТУП}
\subsection{Поясненя якоїсь хрені}

Створення  інформаційних  систем  сьогодні  здійснюється  
на  основі сучасних  методологічних  концепцій,
які  успадкували  найважливіші  ідеї класичних методологій типу
SADT,IDEFO, одночасно збагативши їх новими ідеями.
\subsection{Ну дай ще трохи пояснити}
Найперспективнішим напрямом розвитку програмних
продуктів є клієнт серверна взаємодія в більшості
побудована на основі HTTP протоколу, тобто веб застосування,
що включають в себе ту саму обробку даних, подаючи її
в зручному для користувача вигляді.

Привіт світе

\printImage{cat}{Малюнок котика}
\newpage
\startSection{ІНШИЙ ВСТУП}
\subsection{Інша сабсекція}
\subsubsection{Subsubsection B}
Створення  інформаційних  систем  сьогодні  здійснюється  
на  основі сучасних  методологічних  концепцій,
які  успадкували  найважливіші  ідеї класичних методологій типу
SADT,IDEFO, одночасно збагативши їх новими ідеями.
\subsubsection{Subsubsection B}
Найперспективнішим напрямом розвитку програмних
продуктів є клієнт серверна взаємодія в більшості
побудована на основі HTTP протоколу, тобто веб застосування,
що включають в себе ту саму обробку даних, подаючи її
в зручному для користувача вигляді.
\printImage{cat}{Малюнок котика}
\printImage{cat}{Малюнок котика2}
\printImage{cat}{Малюнок котика3}
\startSection{АХАХАХА}
Створення  інформаційних  систем  сьогодні  здійснюється  
на  основі сучасних  методологічних  концепцій,
які  успадкували  найважливіші  ідеї класичних методологій типу
SADT,IDEFO, одночасно збагативши їх новими ідеями.

Створення  інформаційних  систем  сьогодні  здійснюється  
на  основі сучасних  методологічних  концепцій,
які  успадкували  найважливіші  ідеї класичних методологій типу
SADT,IDEFO, одночасно збагативши їх новими ідеями.

Створення  інформаційних  систем  сьогодні  здійснюється  
на  основі сучасних  методологічних  концепцій,
які  успадкували  найважливіші  ідеї класичних методологій типу
SADT,IDEFO, одночасно збагативши їх новими ідеями.
%\printCodeListing{example.js}{Трохи кодця}
\end{document}