\documentclass{lib/styles/default-style}
\usepackage{lib/docs/title}
\usepackage{lib/docs/task}

\begin{document}
\pagestyle{default-numbered}
\startPageCountFrom{3}

\tableofcontents

\newpage

\unnumberedSection{НАЙКРАЩИЙ ВСТУП}

Створення  інформаційних  систем  сьогодні  здійснюється  
на  основі сучасних  методологічних  концепцій,
які  успадкували  найважливіші  ідеї класичних методологій типу
SADT,IDEFO, одночасно збагативши їх новими ідеями.

\startSection{ВСТУП}
\subsection{Пояснення якоїсь хрені}

Створення  інформаційних  систем  сьогодні  здійснюється  
на  основі сучасних  методологічних  концепцій,
які  успадкували  найважливіші  ідеї класичних методологій типу
SADT,IDEFO, одночасно збагативши їх новими ідеями.
\subsection{Ну дай ще трохи пояснити}
Найперспективнішим напрямом розвитку програмних
продуктів є клієнт серверна взаємодія в більшості
побудована на основі HTTP протоколу, тобто веб застосування,
що включають в себе ту саму обробку даних, подаючи її
в зручному для користувача вигляді.

Привіт світе

\subsubsection{Subsubsection B}
Найперспективнішим напрямом розвитку програмних
продуктів є клієнт серверна взаємодія в більшості
побудована на основі HTTP протоколу, тобто веб застосування,
що включають в себе ту саму обробку даних, подаючи її
в зручному для користувача вигляді.
\printImage{cat}{Малюнок котика}
\printImage{cat}{Малюнок котика2}
\printImage{cat}{Малюнок котика3}
\startSection{АХАХАХА}
Створення  інформаційних  систем  сьогодні  здійснюється  
на  основі сучасних  методологічних  концепцій,
які  успадкували  найважливіші  ідеї класичних методологій типу
SADT,IDEFO, одночасно збагативши їх новими ідеями.

Створення  інформаційних  систем  сьогодні  здійснюється  
на  основі сучасних  методологічних  концепцій,
які  успадкували  найважливіші  ідеї класичних методологій типу
SADT,IDEFO, одночасно збагативши їх новими ідеями.

Створення  інформаційних  систем  сьогодні  здійснюється  
на  основі сучасних  методологічних  концепцій,
які  успадкували  найважливіші  ідеї класичних методологій типу
SADT,IDEFO, одночасно збагативши їх новими ідеями.

Хопанаа \(ax^2+bx+c=0\):
\begin{equation}\label{eq:solv}
 x_{1,2}=\frac{-b\pm\sqrt{b^2-4ac}}{2a}
\end{equation}
 
Зараз подивимось шо це~\eqref{eq:solv}.

\begin{stdout}{Test}
    <html>
     <head>
      <title>PHP Test</title>
     </head>
     <body>
     <?php echo '<p>Hello World</p>'; ?> 
     </body>
     <title>PHP Test</title>
     </head>
     <body>
     <?php echo '<p>Hello World</p>'; ?> 
     </body>
    </html>
\end{stdout}

\begin{code}{Test}
<html>
<head>
 <title>PHP Test</title>
</head>
<body>
<?php echo '<p>Hello World</p>'; ?> 
</body>
<title>PHP Test</title>
</head>
<body>
<?php echo '<p>Hello World</p>'; ?> 
</body>
</html>
\end{code}

\begin{stdout}{Test}
    <html>
     <head>
      <title>PHP Test</title>
     </head>
     <body>
     <?php echo '<p>Hello World</p>'; ?> 
     </body>
     <title>PHP Test</title>
     </head>
     <body>
     <?php echo '<p>Hello World</p>'; ?> 
     </body>
    </html>
\end{stdout}

\end{document}